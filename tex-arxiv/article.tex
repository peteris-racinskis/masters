\documentclass{article}

\usepackage{arxiv}

\usepackage[utf8]{inputenc} % allow utf-8 input
\usepackage[T1]{fontenc}    % use 8-bit T1 fonts
\usepackage{hyperref}       % hyperlinks
\usepackage{url}            % simple URL typesetting
\usepackage{booktabs}       % professional-quality tables
\usepackage{amsfonts}       % blackboard math symbols
\usepackage{nicefrac}       % compact symbols for 1/2, etc.
\usepackage{microtype}      % microtypography
\usepackage{cleveref}       % smart cross-referencing
\usepackage{lipsum}         % Can be removed after putting your text content
\usepackage{graphicx}
\usepackage{natbib}
\usepackage{doi}

\title{A Motion Capture and Imitation Learning-Based Approach to Robot Control}

% Here you can change the date presented in the paper title
%\date{September 9, 1985}
% Or remove it
%\date{}

\author{ \href{https://orcid.org/0000-0002-8956-179X}{\includegraphics[scale=0.06]{orcid.pdf}\hspace{1mm}Peteris Racinskis}\thanks{Robotics and Machine Perception Laboratory} \thanks{Institute of Electronics and Computer Science, Riga, Latvia} \\
	\texttt{peteris.racinskis@edi.lv} \\
	%% examples of more authors
	\And
	\href{https://orcid.org/0000-0001-5203-3347}{\includegraphics[scale=0.06]{orcid.pdf}\hspace{1mm}Janis Arents}\footnote[1]{test} \footnote[2]{test}\\
	\texttt{janis.arents@edi.lv} \\
	\And
	\href{https://orcid.org/0000-0002-5405-0738}{\includegraphics[scale=0.06]{orcid.pdf}\hspace{1mm}Modris Greitans}\footnote[2]{} \\
	\texttt{modris\_greitans@edi.lv} \\
}

% Uncomment to override  the `A preprint' in the header
%\renewcommand{\headeright}{Technical Report}
%\renewcommand{\undertitle}{Technical Report}
\renewcommand{\shorttitle}{\textit{arXiv} Template}

%%% Add PDF metadata to help others organize their library
%%% Once the PDF is generated, you can check the metadata with
%%% $ pdfinfo template.pdf
\hypersetup{
pdftitle={A Motion Capture and Imitation Learning-Based Approach to Robot Control},
pdfsubject={cs.RO},
pdfauthor={Peteris Racinskis, Janis Arents, Modris Greitans},
pdfkeywords={Imitation Learning, Motion capture, Robotics, Artificial neural networks, RNN},
}

\begin{document}
\maketitle

\begin{abstract}
	Imitation Learning is a discipline of Machine Learning primarily concerned with replicating the behaviour of agents that are known to solve a particular task or family of tasks in a demonstration data set. In an industrial robotics context this presents the opportunity to replace explicit programming of behaviour with demonstrations of the task to be performed. Motion capture is one of the methods with which such demonstrations can be collected. It offers lesser model complexity compared to more indirect observation modalities such as visual data, yet requires additional data pre-processing if signals beyond effector position and orientation are relevant to the task at hand. In this paper, an approach for motion capture-based imitation learning and implicit control signal estimation is introduced and evaluated on an object throwing task.
\end{abstract}


% keywords can be removed
\keywords{Imitation Learning \and Motion capture \and Robotics \and Artificial neural networks \and RNN}


\section{Introduction}

asfasgf asg agasg asgasg 
a


\subsection{Motivating use case}
asgasgasgags
asgasgasgags


\subsection{Related Work}
\label{sec:related}
asgasgasgags
asgasgasgags


asgasgasgagsagaga
asg

\section{Materials and Methods}
\label{sec:materials}

See Section \ref{sec:materials}.

\subsection{Data collection}
\label{sec:collection}

bla bla bla

\subsection{Pre-processing, extraction of implicit control signals}
\label{sec:preprocess}

bla bla bla

\subsection{Models}
\label{sec:models}

bla bla bla

\subsection{Visualization and execution}
\label{sec:exec}


bla bla bla

\subsection{Evaluation metrics}

bla bla bla


\section{Results}

bla bla bla


\section{Discussion}

bla bla bla



\section{Examples of citations, figures, tables, references}
\label{sec:others}

\subsection{Citations}
Citations use \verb+natbib+. The documentation may be found at
\begin{center}
	\url{http://mirrors.ctan.org/macros/latex/contrib/natbib/natnotes.pdf}
\end{center}

Here is an example usage of the two main commands (\verb+citet+ and \verb+citep+): Some people thought a thing \citep{kour2014real, hadash2018estimate} but other people thought something else \citep{kour2014fast}. Many people have speculated that if we knew exactly why \citet{kour2014fast} thought this\dots


\bibliographystyle{unsrtnat}
\bibliography{references}  %%% Uncomment this line and comment out the ``thebibliography'' section below to use the external .bib file (using bibtex) .



\end{document}
